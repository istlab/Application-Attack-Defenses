\documentclass[10pt,journal,compsoc]{IEEEtran}

\let\chapter\section

\usepackage[ruled]{algorithm2e}
\renewcommand{\algorithmcfname}{ALGORITHM}
\SetAlFnt{\small}
\SetAlCapFnt{\small}
\SetAlCapNameFnt{\small}
\SetAlCapHSkip{0pt}
\IncMargin{-\parindent}

\usepackage{hyperref}
\usepackage{graphicx}
\usepackage{fancyhdr}
%\usepackage{lastpage}
\usepackage{amsmath}
\usepackage{amscd}
\usepackage{color}
\usepackage{url}
%\usepackage{moreverb}
\usepackage{verbatim}
\usepackage{textcomp}
\usepackage{mathptmx}
\usepackage{dingbat}
\usepackage{pifont}
\usepackage{acronym}%
\usepackage{algpseudocode}
%\usepackage{algorithm}
\usepackage{supertabular}
\usepackage{listings}
\usepackage{threeparttable}
\usepackage{enumitem}
\usepackage{pdflscape}
\usepackage{array}
\usepackage{multirow}
\usepackage{subfigure}
\usepackage{array}
\usepackage{courier}
\newcommand{\subparagraph}{}
\usepackage{titlesec}

\titlespacing*{\section}
{0pt}{1ex}{0.55ex}
\titlespacing*{\subsection}
{0pt}{1ex}{0.55ex}
\titlespacing*{\subsubsection}
{0pt}{1ex}{0.55ex}

\setlength{\footnotesep}{0.28cm}
\setlength{\skip\footins}{0.1cm}

\usepackage{draftwatermark}
\SetWatermarkText{Confidential}
\SetWatermarkScale{0.4}

\usepackage[numbers,sort]{natbib}

\setlist[itemize]{noitemsep, leftmargin=1em}
\setlist[enumerate]{noitemsep, leftmargin=1em}

\lstset{columns=flexible,showstringspaces=false}
\lstset{basicstyle=\footnotesize\ttfamily,breaklines=true}
\lstset{aboveskip=3.3pt,belowskip=3.3pt}

\renewcommand{\algorithmiccomment}[1]{\hfill {\tt //} #1}
\newcommand{\tick}{\ding{52}}
\newcommand{\xmark}{\ding{56}}

\makeatletter
\newcommand{\thickhline}{%
    \noalign {\ifnum 0=`}\fi \hrule height 1pt
    \futurelet \reserved@a \@xhline
}
\newcolumntype{"}{@{\hskip\tabcolsep\vrule width 1pt\hskip\tabcolsep}}
\makeatother

\begin{document}

\title{Defending Against Web Application Attacks: Approaches, Challenges and Implications
\newline\newline
\small{\textbf{Response to Review Comments}}}

\author{
\IEEEauthorblockN{Dimitris Mitropoulos,\IEEEauthorrefmark{1}
Panos Louridas,\IEEEauthorrefmark{2}
Michalis Polychronakis,\IEEEauthorrefmark{3}
and Angelos D. Keromytis\IEEEauthorrefmark{1}}\\
\IEEEauthorblockA{\IEEEauthorrefmark{1}
\scriptsize Department of Computer Scinence,
Columbia University,
\{dimitro, angelos\}@cs.columbia.edu\\
\IEEEauthorrefmark{2}
Department of Management Science and Technology,
Athens University of Economics and Business,
louridas@aueb.gr\\
\IEEEauthorrefmark{3}
Computer Science Department,
Stony Brook University,
mikepo@cs.stonybrook.edu
\vspace{-3mm}
}}

% \markboth{\footnotesize D. Mitropoulos et al.}{\footnotesize Defending Against Web Application Attacks:
% Research Approaches, Challenges and Implications}

\markboth{Response to Review Comments.}%
{Shell \MakeLowercase{\textit{et al.}}: Bare Demo of IEEEtran.cls for Computer Society Journals}

\IEEEtitleabstractindextext{
\begin{abstract}
Some of the most dangerous web attacks, such as Cross-Site
Scripting and {\sc sql} injection, exploit vulnerabilities
in web applications that may accept and process
data of uncertain origin without proper validation or
filtering, allowing the injection and execution of
dynamic or domain-specific language code.
These attacks have been constantly topping the lists of
various security bulletin providers despite the numerous
countermeasures that have been proposed over the past 15 years.
In this paper, we provide an analysis on
various defense mechanisms against web code injection attacks.
We propose a model that highlights the key weaknesses enabling these attacks,
and that provides a common perspective for studying the available defenses.
We then categorize and analyze a set of 35 previously proposed defenses 
based on their accuracy, performance,
deployment, security, and availability characteristics.
Detection accuracy is of particular importance,
as our findings show that many defense
mechanisms have been tested in a poor manner. In addition, we observe
that some mechanisms can be bypassed by attackers with knowledge
of how the mechanisms work. Finally, we discuss the results
of our analysis, with emphasis on factors that may hinder
the widespread adoption of defenses in practice.
\vspace{-3mm}
\end{abstract}

\begin{IEEEkeywords}
Web Application Security, Protection Mechanisms, Exploitation Models, Software Testing,
SQL Injection, XSS.
\vspace{-1mm}
\end{IEEEkeywords}}

\maketitle

\IEEEdisplaynontitleabstractindextext

\IEEEpeerreviewmaketitle

\section{General Remarks}
\label{sec:rem}

We are thankful to the reviewers for their thorough reviews.
We have revised our paper based on their suggestions and comments.
All additions and changes are highlighted in yellow
in the revised manuscript.
Number wise answers to their specific
comments / suggestions / queries are as follows.

\section{AE Comments}
\label{sec:ae}

{\bf Comment} ``The second and third reviewers have pointed out problems in missing related work and
descriptions that need to polished. Please address them carefully in the next revision
for the next round of review."\\

\noindent
{\bf Response:}
Lorem Ipsum. Lorem Ipsum. Lorem Ipsum. Lorem Ipsum. Lorem Ipsum. Lorem Ipsum. Lorem Ipsum. Lorem Ipsum. Lorem Ipsum. Lorem Ipsum. Lorem Ipsum. Lorem Ipsum. Lorem Ipsum. Lorem Ipsum.\\

\section{Reviewer \#1}
\label{sec:r1}

{\bf Comment 1:} ``One downside of this work is that it was not clear from the
description what constitutes a defense against a web application
attack. The authors do say that vulnerability analysis techniques are
out-of-bounds (which I agree with), but what about related
technologies such as Web Application Firewalls? Where does one draw
the line? What are the criteria to evaluate new techniques to see if
they fit into your defenses (so that the authors can evaluate using
your techniques)?"\\

\noindent
{\bf Response:}
Indeed,
an application firewall is actually a defense
mechanism that fits in our study.
Besides,
in reference~\cite{DPJV06}
the authors indicate three mechanisms~\cite{BK04,PB05,HO05b}
that we include in our study as
application firewalls.
Such mechanisms though,
can be seen as host-based application firewalls
which are different from the network-based
application firewalls.
Mechanisms coming from the later category
operate directly at the application
layer of the protocol stack and
in most cases they are modules
of the web server
(e.g. {\tt modsecurity})\footnote{\url{https://www.modsecurity.org/}}.
Even though we do not include such mechanisms
(note also that {\tt modsecurity} is
not a research approach),
they can be analyzed based on our criteria.\\
% note: I do not know if we must update the text
% of the paper to address the above comment.

\noindent
{\bf Comment 2:} ``Also, I felt that the focus on studying ``popular" defenses (those with
greater than 20 citations or published recently) ignores one of the	
main benefits of a survey paper: to help the community discover
overlooked yet quality research. I appreciate that surveying the 35
papers is a good contribution, but I wonder if the results of the
evaluation would be more impactful (in terms of finding overlooked
research) if the survey was more inclusive."\\

\noindent
{\bf Response:}
Lorem Ipsum. Lorem Ipsum. Lorem Ipsum. Lorem Ipsum. Lorem Ipsum. Lorem Ipsum. Lorem Ipsum. Lorem Ipsum. Lorem Ipsum. Lorem Ipsum. Lorem Ipsum. Lorem Ipsum. Lorem Ipsum. Lorem Ipsum.\\

\noindent
{\bf Comment 3:} ``Some minor comments for the authors:
\begin{enumerate}
\item Introduction: ``didn't" $\rightarrow$ did not
\item Introduction: ``haven't" $\rightarrow$ have not
\item Section 2: PHP Object Injection vulnerabilities do not have to be
 ``part of cookies". These could come any user input, not necessarily
 cookies (similar to other code injection vulnerabilities).
\item Section 4.5 ``canny attacks" and ``cannier attacker" should probably
 be ``savvy" or a different word.
\item Section 5.1.2 it is not clear from your description that the CSP
 policy is communicated via HTTP header. This may not be clear to
 those readers that aren't familiar with CSP.
\item Section 5.1.2 ``XXS" $\rightarrow$ ``XSS".
\item Section 5.2.1 ``maybe" $\rightarrow$ ``may be"".
\end{enumerate}

\noindent
{\bf Response:}
We have updated the text accordingly.\\

\section{Reviewer \#2}
\label{sec:r2}

{\bf Comment 1:} ``Τhe analysis on the accuracy and performance overhead is not reasonable. The paper presents metrics that were originally reported in other papers, which were evaluated under different assumptions and settings. Simply listing the numbers found in those papers in a table might be misleading. However, the reviewer acknowledges that it is very difficult to have access to those systems for a fair comparison, and agrees with the recommendation on improving the availability of the source code and dataset in future research."\\

\noindent
{\bf Response:}
Lorem Ipsum. Lorem Ipsum. Lorem Ipsum. Lorem Ipsum. Lorem Ipsum. Lorem Ipsum. Lorem Ipsum. Lorem Ipsum. Lorem Ipsum. Lorem Ipsum. Lorem Ipsum. Lorem Ipsum. Lorem Ipsum. Lorem Ipsum.\\

\noindent
{\bf Comment 2:} ``Admittedly, transition from scientific research to real product
is challenging due to availability, usability, performance and deployment issues.
Except for that, the paper does not thorouly discuss why many previous works
cannot stop attacks against web applications, nor provide useful guidance on
how future research could improve upon existing works. Are there any fundamental
limitations in previous works that make them not suitable for defending against
some specific (if not all) attacks? Can we integrate multiple solutions to build
stronger defense mechanisms?"\\

\noindent
{\bf Response:}
Lorem Ipsum. Lorem Ipsum. Lorem Ipsum. Lorem Ipsum. Lorem Ipsum. Lorem Ipsum. Lorem Ipsum. Lorem Ipsum. Lorem Ipsum. Lorem Ipsum. Lorem Ipsum. Lorem Ipsum. Lorem Ipsum. Lorem Ipsum".\\

\noindent
{\bf Comment 3:} ``Some recent works in the (or related) 
ield are not discussed in this paper.
For example, information flow control solutions and
proposals that confine or isolate JavaScript could
be used to defend against code injection attacks.
Examples are: {\sc jsf}low: Tracking information
flow in JavaScript and its {\sc api}s ({\sc sac}'14).
Protecting users by confining JavaScript with {\sc cowl} ({\sc osdi} '14).
Runtime monitoring and formal analysis of information fows in Chromium ({\sc ndss} '15)."\\

\noindent
{\bf Response:}
Lorem Ipsum. Lorem Ipsum. Lorem Ipsum. Lorem Ipsum. Lorem Ipsum. Lorem Ipsum. Lorem Ipsum. Lorem Ipsum. Lorem Ipsum. Lorem Ipsum. Lorem Ipsum. Lorem Ipsum. Lorem Ipsum. Lorem Ipsum.\\

\section{Reviewer \#3}
\label{sec:r3}

{\bf Comment 1:} ``The biggest omission that I see, however, is that the
exploitation model does not consider inter-frame attacks, where an attacker
e.g., lures victims to framed vulnerable sites and performs cross-frame code
injection.  Granted, this is a relatively newer update to the pantheon of
web application attacks, but given the rising prevalence of client-side
web application code it's in my estimation an important one.  And, postMessage
security research dates back to at least the late 2000s, so it's actually not
*that* new."\\

\noindent
{\bf Response:}
Lorem Ipsum. Lorem Ipsum. Lorem Ipsum. Lorem Ipsum. Lorem Ipsum. Lorem Ipsum. Lorem Ipsum. Lorem Ipsum. Lorem Ipsum. Lorem Ipsum. Lorem Ipsum. Lorem Ipsum. Lorem Ipsum. Lorem Ipsum".\\

\noindent
{\bf Comment 2:} ``Similarly, {\sc xss} is introduced as an attack that has two forms, reflected and
persistent.  However, even though it's mentioned later on, how would DOM-based
{\sc xss} fit into this breakdown?  I don't believe it does (neatly), and am of the
view that it constitutes a third category since it can be performed on-demand
and purely client-side across frames without persistence, combining aspects of
both of the above categories."\\

\noindent
{\bf Response:}
Lorem Ipsum. Lorem Ipsum. Lorem Ipsum. Lorem Ipsum. Lorem Ipsum. Lorem Ipsum. Lorem Ipsum. Lorem Ipsum. Lorem Ipsum. Lorem Ipsum. Lorem Ipsum. Lorem Ipsum. Lorem Ipsum. Lorem Ipsum".\\

\noindent
{\bf Comment 3:} ``Aside from those issues, one could take issue with a lack of coverage of some
(in my opinion) important approaches that hew towards the
secure-by-construction ideology. For instance, there is work from the {\sc pl}
community (e.g., Ur/Web) that makes certain attacks impossible by design.
There is other {\sc pl} work I didn't see mentioned that also takes a
secure-by-design approach (e.g., {\sc jif}/{\sc sif}). There is work from the systems
community focused on {\sc difc} (e.g., Hails, Aeolus).  I anticipate that these
wouldn't rate highly on the ease-of-use metrics, but it would have been nice
to see them mentioned.  I did notice that static analyses were specifically
called out as not being in-scope, but other work that was mentioned does
require application modification (e.g., {\sc js} confinement).  So, while I'm sure
there's a line one should draw, it's not clearly defined here and some of the
above could, in the current presentation, be considered missing."\\

\noindent
{\bf Response:}
Lorem Ipsum. Lorem Ipsum. Lorem Ipsum. Lorem Ipsum. Lorem Ipsum. Lorem Ipsum. Lorem Ipsum. Lorem Ipsum. Lorem Ipsum. Lorem Ipsum. Lorem Ipsum. Lorem Ipsum. Lorem Ipsum. Lorem Ipsum".\\

\bibliographystyle{IEEEtran}
\bibliography{comments}

\end{document} 

