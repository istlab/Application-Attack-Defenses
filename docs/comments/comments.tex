\documentclass[10pt,journal,compsoc]{IEEEtran}

\let\chapter\section

\usepackage[ruled]{algorithm2e}
\renewcommand{\algorithmcfname}{ALGORITHM}
\SetAlFnt{\small}
\SetAlCapFnt{\small}
\SetAlCapNameFnt{\small}
\SetAlCapHSkip{0pt}
\IncMargin{-\parindent}

\usepackage{hyperref}
\usepackage{graphicx}
\usepackage{fancyhdr}
%\usepackage{lastpage}
\usepackage{amsmath}
\usepackage{amscd}
\usepackage{color}
\usepackage{url}
%\usepackage{moreverb}
\usepackage{verbatim}
\usepackage{textcomp}
\usepackage{mathptmx}
\usepackage{dingbat}
\usepackage{pifont}
\usepackage{acronym}%
\usepackage{algpseudocode}
%\usepackage{algorithm}
\usepackage{supertabular}
\usepackage{listings}
\usepackage{threeparttable}
\usepackage{enumitem}
\usepackage{pdflscape}
\usepackage{array}
\usepackage{multirow}
\usepackage{subfigure}
\usepackage{array}
\usepackage{courier}
\newcommand{\subparagraph}{}
\usepackage{titlesec}

\titlespacing*{\section}
{0pt}{1ex}{0.55ex}
\titlespacing*{\subsection}
{0pt}{1ex}{0.55ex}
\titlespacing*{\subsubsection}
{0pt}{1ex}{0.55ex}

\setlength{\footnotesep}{0.28cm}
\setlength{\skip\footins}{0.1cm}

\usepackage{draftwatermark}
\SetWatermarkText{Confidential}
\SetWatermarkScale{0.4}

\usepackage[numbers,sort]{natbib}

\setlist[itemize]{noitemsep, leftmargin=1em}
\setlist[enumerate]{noitemsep, leftmargin=1em}

\lstset{columns=flexible,showstringspaces=false}
\lstset{basicstyle=\footnotesize\ttfamily,breaklines=true}
\lstset{aboveskip=3.3pt,belowskip=3.3pt}

\renewcommand{\algorithmiccomment}[1]{\hfill {\tt //} #1}
\newcommand{\tick}{\ding{52}}
\newcommand{\xmark}{\ding{56}}

\makeatletter
\newcommand{\thickhline}{%
    \noalign {\ifnum 0=`}\fi \hrule height 1pt
    \futurelet \reserved@a \@xhline
}
\newcolumntype{"}{@{\hskip\tabcolsep\vrule width 1pt\hskip\tabcolsep}}
\makeatother

\begin{document}

\title{Defending Against Web Application Attacks: Approaches, Challenges and Implications
\newline\newline
\small{\textbf{Response to Review Comments}}}

\author{
\IEEEauthorblockN{Dimitris Mitropoulos,\IEEEauthorrefmark{1}
Panos Louridas,\IEEEauthorrefmark{2}
Michalis Polychronakis,\IEEEauthorrefmark{3}
and Angelos D. Keromytis\IEEEauthorrefmark{1}}\\
\IEEEauthorblockA{\IEEEauthorrefmark{1}
\scriptsize Department of Computer Scinence,
Columbia University,
\{dimitro, angelos\}@cs.columbia.edu\\
\IEEEauthorrefmark{2}
Department of Management Science and Technology,
Athens University of Economics and Business,
louridas@aueb.gr\\
\IEEEauthorrefmark{3}
Computer Science Department,
Stony Brook University,
mikepo@cs.stonybrook.edu
\vspace{-3mm}
}}

% \markboth{\footnotesize D. Mitropoulos et al.}{\footnotesize Defending Against Web Application Attacks:
% Research Approaches, Challenges and Implications}

\markboth{Mitropoulos et al.}%
{Shell \MakeLowercase{\textit{et al.}}: Bare Demo of IEEEtran.cls for Computer Society Journals}

\IEEEtitleabstractindextext{
\begin{abstract}
Some of the most dangerous web attacks, such as Cross-Site
Scripting and {\sc sql} injection, exploit vulnerabilities
in web applications that may accept and process
data of uncertain origin without proper validation or
filtering, allowing the injection and execution of
dynamic or domain-specific language code.
These attacks have been constantly topping the lists of
various security bulletin providers despite the numerous
countermeasures that have been proposed over the past 15 years.
In this paper, we provide an analysis on
various defense mechanisms against web code injection attacks.
We propose a model that highlights the key weaknesses enabling these attacks,
and that provides a common perspective for studying the available defenses.
We then categorize and analyze a set of 35 previously proposed defenses 
based on their accuracy, performance,
deployment, security, and availability characteristics.
Detection accuracy is of particular importance,
as our findings show that many defense
mechanisms have been tested in a poor manner. In addition, we observe
that some mechanisms can be bypassed by attackers with knowledge
of how the mechanisms work. Finally, we discuss the results
of our analysis, with emphasis on factors that may hinder
the widespread adoption of defenses in practice.
\vspace{-3mm}
\end{abstract}

\begin{IEEEkeywords}
Web Application Security, Protection Mechanisms, Exploitation Models, Software Testing,
SQL Injection, XSS.
\vspace{-1mm}
\end{IEEEkeywords}}

\maketitle

\IEEEdisplaynontitleabstractindextext

\IEEEpeerreviewmaketitle

\section{General Remarks}
\label{sec:rem}

We are thankful to the reviewers for their thorough reviews.
We have revised our research paper based on their suggestions and comments.
All additions and changes are highlighted in yellow
in the revised manuscript.
Number wise answers to their specific
comments / suggestions / queries are as follows.

\section{AE Comments}
\label{sec:ae}

{\bf Comment} ``Lorem Ipsum. Lorem Ipsum. Lorem Ipsum. Lorem Ipsum. Lorem Ipsum. Lorem Ipsum. Lorem Ipsum. Lorem Ipsum. Lorem Ipsum. Lorem Ipsum. Lorem Ipsum. Lorem Ipsum.".\\

\noindent
{\bf Response:}
Lorem Ipsum. Lorem Ipsum. Lorem Ipsum. Lorem Ipsum. Lorem Ipsum. Lorem Ipsum. Lorem Ipsum. Lorem Ipsum. Lorem Ipsum. Lorem Ipsum. Lorem Ipsum. Lorem Ipsum. Lorem Ipsum. Lorem Ipsum".\\

\section{Reviewer \#1}
\label{sec:r1}

{\bf Comment 1:} ``Lorem Ipsum. Lorem Ipsum. Lorem Ipsum. Lorem Ipsum. Lorem Ipsum. Lorem Ipsum. Lorem Ipsum. Lorem Ipsum. Lorem Ipsum. Lorem Ipsum. Lorem Ipsum. Lorem Ipsum. Lorem Ipsum. Lorem Ipsum".\\

\noindent
{\bf Response:}
Lorem Ipsum. Lorem Ipsum. Lorem Ipsum. Lorem Ipsum. Lorem Ipsum. Lorem Ipsum. Lorem Ipsum. Lorem Ipsum. Lorem Ipsum. Lorem Ipsum. Lorem Ipsum. Lorem Ipsum. Lorem Ipsum. Lorem Ipsum".\\

\noindent
{\bf Comment 2:} ``Lorem Ipsum. Lorem Ipsum. Lorem Ipsum. Lorem Ipsum. Lorem Ipsum. Lorem Ipsum. Lorem Ipsum. Lorem Ipsum. Lorem Ipsum".\\

\noindent
{\bf Response:}
Lorem Ipsum. Lorem Ipsum. Lorem Ipsum. Lorem Ipsum. Lorem Ipsum. Lorem Ipsum. Lorem Ipsum. Lorem Ipsum. Lorem Ipsum. Lorem Ipsum. Lorem Ipsum. Lorem Ipsum. Lorem Ipsum. Lorem Ipsum".\\

\bibliographystyle{IEEEtran}
\bibliography{comments}

\end{document} 

