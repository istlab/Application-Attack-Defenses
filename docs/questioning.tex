\documentclass[conference]{IEEEtran}

\usepackage{hyperref}
%\usepackage{graphicx}
\usepackage{fancyhdr}
%\usepackage{lastpage}
\usepackage{amsmath}
\usepackage{amscd}
\usepackage{color}
\usepackage{url}
%\usepackage{moreverb}
\usepackage{verbatim}
\usepackage{textcomp}
\usepackage{mathptmx}
\usepackage{dingbat}
\usepackage{pifont}
\usepackage{acronym}
\usepackage{algpseudocode}
\usepackage{algorithm}
\usepackage{supertabular}
\usepackage{listings}
\usepackage{threeparttable}
\usepackage{pdflscape}
\usepackage{array}
\usepackage{multirow}
\usepackage{subfigure}

\usepackage[numbers,sort]{natbib}

\renewcommand{\algorithmiccomment}[1]{\hfill {\tt //} #1}
\newcommand{\tick}{\ding{52}}
\newcommand{\xmark}{\ding{56}}

\date{}
\begin{document}

\author{
\IEEEauthorblockN{Dimitris Mitropoulos,\IEEEauthorrefmark{1}
Panos Louridas,\IEEEauthorrefmark{2} and Angelos Keromytis\IEEEauthorrefmark{1}}
\IEEEauthorblockA{\IEEEauthorrefmark{1}
Network Security Lab\\
Department of Computer Scinence\\
Columbia University\\
\{dimitro, angelos\}@cs.columbia.edu\\
\IEEEauthorrefmark{2}
Software Engineering and Security Lab\\
Department of Management Science and Technology\\
Athens University of Economics and Business\\
louridas@aueb.gr
}}

\title{Questioning how Researchers Develop and Present Intrusion Detection Systems}

\maketitle
\begin{abstract}
In this paper we examine how intrusion-detection
systems developed to counter a specific category
of application attacks are developed and presented to
the research community.
\end{abstract}

\begin{IEEEkeywords}
Security and Protection, Intrusion-Detection Systems, Code injection Attacks, Cyber Security Experiments, Effectivenes, Efficiency.
\end{IEEEkeywords}

\IEEEpeerreviewmaketitle

\section{Introduction}

Introduction~\cite{I05}. Also see~\cite{A00}.

Requirements:
\begin{itemize}
	\item {\bf Flexibility} We check if an approach
can be adjusted in order to detect different {\sc cia} categories.
	\item {\bf Ease of use} We examine if the mechanism that
implements an approach is practical and can be easily adopted
by security experts.
	\item {\bf Effectiveness} As long as we examine security
mechanisms that detect either attacks or defects,
the non-existence of false positive and negative alarms
is a reasonable requirement.
	\item {\bf Efficiency} Finally, we examine
the computational overhead of the mechanisms that affect the experience of
a web user who actually uses the protected application.
	\item {\bf Security} How resilient is the system to
attacks made to to circumvent it.
\end{itemize}
All the aforementioned requirements are considered critical
when building security mechanisms that protect
applications~\cite{A01,A00}.

\section{Covered Area}

We decided to narrow down our
research to countermeasures developed to
detect attacks that target applications. Such attacks include
buffer overflow attacks~\cite{K11}, {\sc sql} injection
attacks~\cite{RL12b}, cross-site scripting ({\sc xss})
attacks~\cite{SG07}, cross-site request forgery ({\sc csrf})
attacks~\cite{LZRL09} and others.
Such attacks top the vulnerability lists of numerous bulletin providers for several
years.\footnote{\url{http://www.sans.org/top-cyber-security-risks/}, \url{http://cwe.mitre.org/top25/}}
Consider the {\sc owasp}\footnote{\url{https://www.owasp.org/index.php/Category:OWASP_Top_Ten_Project}}
(Open Web Application Security Project)
Top Ten project whose main goal is to raise awareness about
web application security by identifying some of the most critical risks facing
organizations which is referenced by numerous researchers.
In its three consecutive Top Ten lists (2007, 2010, 2013), different
source code-driven injection attacks dominate the top five positions.
This indicates that
apart from the fact that malicious users find new ways to bypass
defense mechanisms by using a variety of techniques despite the numerous
countermeasures that are being introduced.
Note that the number of systems developed to counter {\sc sql}
injection attacks until 2006 was more than twenty~\cite{HVO06}.
Since then the number has doubled.
In our research we focus on the top {\sc tba} systems that counter
application attacks, in terms of citations.

\section{Analysis}

\begin{table*}
%\tbl{Performance of System Prototype. Time is Measured in $\mu$s.}{
\centering
    \begin{threeparttable}
    \begin{small}
\scalebox{0.93}{
    \begin{tabular}{l|c|cc|c}
    \hline
    \bf{Approach}
	& \bf{Mechanism}
    & \multicolumn{2}{|c|}{Requirements}
	& \bf{Attack Vector} \\
	&& \bf{FP/FN}
	& \bf{Computational Overhead}\tnote{4} & \\
    \hline
	\multirow{13}{*}{Runtime Tainting}
	&   {\it Kiriansky et al.~\cite{KBA02}} & {\bf ?},{\bf ?} & $\sim$1\% & {\it binary code} \\
	&   {\it {\sc cfi}~\cite{ABEL05}} & {\bf ?},{\bf ?} & 0.09--26.78\% & {\it binary code} \\
	&  	{\it SigFree}~\cite{WPLZ10} & \xmark,\xmark & 10\% & {\it binary code} \\
	&  	{\sc csse}~\cite{PB05} & \xmark,\xmark & 2--10\% & {\sc sql} \\
	&  	{\it {\sc lift}}~\cite{QWLKZW06} & \tick,\tick & 6.2\% & {\it binary code} \\ 
	&  	{\it Haldar et al.}~\cite{HCF05} & {\bf ?},{\bf ?} & \xmark & {\sc sql} \\ 
	&  	{\it SecuriFly}~\cite{MLL05} & \xmark,\tick & 9--125\% & {\sc sql}, {\sc xss} \\ 
	&  	{\it Xu et al.}~\cite{XBS06} & \tick,\xmark & 76\% & {\sc sql}, {\sc xss} \\ 
    &  	{\it {\sc wasc}}~\cite{NLC07} & \xmark,\xmark & 30\% & {\sc sql}, {\sc xss} \\
	&  	{\it {\sc php} Aspis}~\cite{PMP11} & \xmark,\xmark & 2.2$\times$ & {\sc sql}, {\sc xss}, {\sc php} \\
	&  	{\it Vogt et al.}~\cite{VFJKKV07} & \xmark,{\bf ?} & {\bf ?} & {\sc xss} \\
	& 	{\it Stock et al.}~\cite{SLMS14} & 0.09\%,\xmark & 7-17\% & {\sc xss} \\
	%& {\it Lekies et al.}~\cite{LSJ13} & & & \\
	\hline
	\hline      
	\multirow{3}{*}{{\sc isr}}
	&  {\it {\sc sql}rand}~\cite{BK04} & {\bf ?},{\bf ?} & 6.5{\it ms} & {\sc sql} \\ 
	&  {\it Noncespaces}~\cite{GC09} & \tick,\tick &  10.3\% & {\sc xss} \\ 
    &  {\it x{\sc js}}~\cite{APKLM10} & \tick,\tick & 1.6--40{\it ms} & {\sc xss} \\
	\hline
	\hline    
	\multirow{14}{*}{Policy Enforcement}
	&  {\it {\sc dsi}}~\cite{NSS06} & \xmark,\xmark & 1.85\% & {\sc xss} \\ 
	&  {\it BrowserShield}~\cite{RDWDE07} & \tick,\tick & 8\% & {\sc xss} \\ 
	&  {\it Blueprint}~\cite{LV09} & \tick/\tick & 13.6\% & {\sc xss} \\ 
	&  {\it WebJail}~\cite{VDDPJ11} & {\bf ?},{\bf ?} & $\sim$7ms & {\sc xss} \\ 
	&  {\it CoreScript}~\cite{YCIS07} & {\bf ?},{\bf ?} &  {\bf ?} & {\sc xss} \\ 
	&  {\it ConScript}~\cite{ML10} & {\bf ?},{\bf ?} & 7\% & {\sc xss} \\
	&  {\it {\sc js}and}~\cite{AVBPDP12} & {\bf ?},{\bf ?} & up to 365\% & {\sc xss}\\
	&  {\it {\sc met}}~\cite{ELX07} & {\bf ?},{\bf ?} &  {\bf ?} & {\sc xss} \\ 
    &  {\it {\sc beep}}~\cite{TNH07}  & \tick,\tick & 14.4\% & {\sc xss} \\
	&  {\it {\sc csp}} & {\bf ?},{\bf ?} & {\bf ?} & {\sc xss} \\ 
    &  {\it Google Caja} & {\bf ?},{\bf ?} & {\bf ?} & {\sc xss} \\
    &  {\it Phung et al.}~\cite{PSC09} & \xmark,{\bf ?} & 5.37\% & {\sc xss} \\
    &  {\it TreeHouse}~\cite{IW12} & {\bf ?},{\bf ?} & 757–--1218ms & {\sc xss} \\
    &  {\it Dhawan et al.}~\cite{DSG12} & \xmark,{\bf ?} & 16---114\% & {\sc xss} \\
	\hline
	\hline  
        \multirow{12}{*}{Training}
	&   {\it {\sc amnesia}}~\cite{HO05,HO06,HO05b} & \tick,\tick & {\bf ?} & {\sc sql} \\ 
	&   {\it {\sc didafit}}~\cite{LLW02} & \xmark,\xmark & {\bf ?} & {\sc sql} \\
	&   {\it Valeur et al.}~\cite{VMV05} & 0.37\%,{\bf ?} & 1{\it ms} & {\sc sql} \\
	& 	{\it {\sc sqlg}uard}~\cite{BWS05} & {\bf ?},{\bf ?} & 3\% & {\sc sql} \\
	& 	{\it Diglossia}~\cite{SMS13} & \xmark,\xmark  & 13\% & {\sc sql}, No{\sc sql} \\
	& 	{\it {\sc sd}river}~\cite{MS09,MKS09} & \tick,\tick & 39\% & {\sc sql}, {\sc xp}ath \\
	& 	{\it Laranjeiro et al.}~\cite{LVM09,ALVM09,LVM10} & \xmark,\xmark  & \xmark & {\sc sql}, {\sc xp}ath \\
	& 	{\it Mattos et al.}~\cite{MSM13} & \tick,{\bf ?} &  {\bf ?} & {\sc xml}, {\sc xp}ath \\
	& 	{\it {\sc sm}ask}~\cite{JB07} & \xmark,\xmark & {\bf ?} & {\sc sql}, {\sc xss} \\
	& 	{\it {\sc swap}}~\cite{WPLKK09} & \tick,\tick & $\sim$180\% & {\sc xss} \\ 
    & 	{\it {\sc xssds}}~\cite{JEP08}  & \xmark,\xmark & {\bf ?} &  {\sc xss} \\
    & 	{\it {\sc xss-guard}}~\cite{BV08} & {\bf ?},\xmark & 5---24\% & {\sc xss} \\
	\hline
    \end{tabular}}
    \begin{tablenotes}
	\begin{footnotesize}
       \item[1] {\it Flexibility} indicates if the approach can be adjusted
	in order to detect different {\sc cia} categories.
       \item[4] {\it User Experience} is affected if the mechanism suffers
	from runtime overhead.
	\end{footnotesize}
    \end{tablenotes}
    \caption{Comparison summary of mechanisms developed to counter application attacks.}
    \label{tab:comp2}
    \end{small}
    \end{threeparttable}
\end{table*}

\bibliographystyle{IEEEtran}
\bibliography{questioning}

\end{document} 

